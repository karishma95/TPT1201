\documentclass[12pt]{article}
\renewcommand{\baselinestretch}{1.3} 
\title{\textbf{Biometric Security In Cloud Computing}} 
\date{September 2015}
\author{
Author\\
\textbf{Karishma Patel}\\
Faculty of Computing \& Informatics\\
Multimedia University, Malaysia} 
\begin{document}

\begin{titlepage}
\maketitle
\end{titlepage}

\textbf{\large Executive Summary of Research Proposal} \newline \newline \newline
Biometrics refers to the utilization of a unique physiological attributes to distinguish a person. Various biometric attributes have been created and are utilized to confirm the individual's personality. The thought is to utilize the extraordinary attributes of an individual to distinguish a person. Cloud computing is one of the rising advancements, that takes system clients to the following level. Cloud is an innovation where assets are paid per utilization as opposed to claimed. One of the greatest difficulties in this innovation is security. Despite the fact that clients utilization administration supplier's assets, there is an incredible level of hesitance from clients' end due to huge security dangers pressed with this innovation. Research in this center has given various answers for defeat these security obstructions. Each of these has its own particular upsides and downsides. The main objective of this model is to provide a more secured cloud with fingerprints and numbers. This article proposes another model of a security framework where in clients are to give numerous biometric fingerprints during enrolment for an administration. These formats are put away at the cloud supplier's end. The clients are validated in view of these unique mark layouts which must be given in the request of irregular numbers that are created without fail. Both fingerprints layouts and pictures gave each time are scrambled to improved security. \newpage

\textbf{\large  Introduction} \newline \newline \newline
Biometric validation utilizes human attributes like fingerprints, tongue impressions, iris and face recognitions that are one of a kind to every person and in this way differentiating clients. A general methodology in a biometric framework is to store all caught biometric pictures in the enlistment stage, and validation is done utilizing a coordinating procedure. This system, without a doubt experiences security shortcomings. Defenseless capacity may prompt an aggressor taking biometric formats and mimicking the genuine client. The stolen biometric data may trade off different frameworks. Cloud computing refers to an on-demand, self-administration Internet base that empowers clients to get to registering resources from any place and whenever. The administrations offered by a cloud can be classified into Software as a Service, Platform as a Service, Infrastructure as a Service, and Storage as a Service and so on. Sending of a cloud falls into three sorts, as follows open, private and group cloud. In an open cloud, assets are interested in the overall population over the Internet. A private cloud base is worked for a solitary association. At the point when the assets are imparted among associations to normal concerns, then it turns into a group cloud. Multitenancy, gigantic adaptability, versatility are the essential qualities of a cloud innovation. Cloud calculations are operational on unfixed hubs in a system, prompting information misfortune and protection issues. In this article, another security model has been recommended that uses various fingerprints and arbitrary numbers as validation devices. \newpage

\textbf{\large Justification of Research} \newline \newline 
Multi finger security model is a system where clients, during enlistment can enrol with three finger templates of their decision and assign a solitary digit number for each of these three fingers. The proposed security model has an edge over different models that give single unique mark framework. The reason is that, once a intruder obtains entrance to a unique mark format, he can claim to be a verified client. In any case, in a different unique mark framework, regardless of the fact that an even if an intruder manages to lacerate a stored template, still number labelled to each of the finger stays hidden. \newline \newline 

\textbf{\large Research Objectives} \newline \newline
1.  To provide a more secured cloud with fingerprints and numbers.\newline \newline
2.  To prevent data leakage to random clients that may control the information for their own utilization. \newline \newline

\textbf{\large Literature Review} \newline \newline
Combining biometric identification and cloud computing with the aim of a safe cloud calculation has never been a new innovation. Contingent upon the biometric strategy utilized, a mixed bag of security vulnerability are conceivable when biometric is utilized for security purposes.  \newpage

\textbf{\large Research Methodology} \newline \newline
The research method used is theoretical. Security is required in the cloud computing business to keep up users' trust as the users are believing the administration suppliers to keep up the confidential informations. By having a mobile unique finger impression scanner, its less demanding for clients to not need to recall all the long passwords just to secure their data. To be on the more secure side, clients are obliged to give fingerprints to three fingers. This model can be evaluated at three stages : \newline \newline
\textbf {Enlistment stage : } At the point when a client enlists for an administration, he enrols with three finger characteristics of his decision. The client then allocates three single digit quantities of his decision. All the three inputs, finger print pictures, three single digit numbers and mapping of numbers to fingers are all encoded and stored at the administration supplier's end. \newline \newline
\textbf {Access stage : } At the point when an access is made to the cloud, client gives finger impressions of these three enlisted fingerprints. The impressions' request is taking into account three digit arbitrary numbers produced. \newline \newline
\textbf {Matching stage : } In this stage, a genuine client is approved and a intruder is nullified. Regardless of the fact that the stored layouts are hacked, the request of giving the impressions differs the irregular number produced. Along these lines by method for experimentation, if a hacker tries with distinctive stages, access will be denied after three continuous wrong attempts. The client needs to re-set the numbering that was assigned before. This stage likewise incorporates a strategy for reassignment of a biometric format alongside numbers and mappings when the current one, thought to be traded off after three back to back wrong attempts.  \newpage

\textbf{\large Bibliography} \newline \newline
1.  http://blog.m2sys.com/category/cloud-computing/ \newline \newline
2.  http://www.biometricupdate.com/service-directory/fingerprint-recognition \newline \newline
3.  http://www.planetbiometrics.com/article-details/i/2332/ \newline \newline
4.  https://books.google.com.my/books\newline \newline
5.  Ravikiran Peelukhana, Shanthi Bala.P, Aghila.G, Department of       
    Computer Science, Pondicherry University, Pudhucherri, India –    
    Securing Virtual Images Using Blind Authentication Protocol-     
    International Journal of Engineering science and Technology 
\end{document}