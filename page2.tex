\documentclass{report}
\usepackage[utf8]{inputenc}
 
\setlength{\arrayrulewidth}{0.08mm}
\setlength{\tabcolsep}{20pt}
\renewcommand{\arraystretch}{2.2}
 
\begin{document}
\begin{tabular}{ |p{4cm}|p{6cm}| }
\hline
\textbf {\large Methodology} & {\large This research has been done quantitatively for the protected computation techniques for the biometric data. For the fingerprint and iris presentation, there are a couple of figurings indicated to demonstrate the security of the biometric information. }   \\
\hline
\textbf {\large Conclusion} & {\large One vital issue is likewise biometric precision of the framework. However utilizing rearranged representations with settled length and basic separation. All the more further exertion should be made to guarantee security in really sent biometric distinguishing proof.} \\
\hline
\textbf {\large What I learnt and Possible Extension / Future Work} & {\large What i have learnt from this article is that biometric data are data can be extra secured. Much exertion has likewise been done to utilize the conceivable outcomes offered by cloud computing in secure computation. An examination around there to enhance the effectiveness as far as memory necessities and computation time. This may be reasonable in the nearing years. } \\
\hline
\textbf {\large References} & {\large S. Prabhakar, S.Pankanti and AK Jain,"Security and Privacy Concerns," IEEE Sec.Privacy, Vol.1, No.2, pp. 33-42, 2003.} \\
\hline
\end{tabular}
\end{document}
\clearpage