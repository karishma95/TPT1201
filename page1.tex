\documentclass{report}

\usepackage[utf8]{inputenc}
 
\setlength{\arrayrulewidth}{0.08mm}
\setlength{\tabcolsep}{22pt}
\renewcommand{\arraystretch}{2.2}
 
\begin{document}

\begin{tabular}{ |p{4cm}|p{6cm}| }
\hline
\textbf {\large Paper Title} & { \large Privacy - Preserving Biometric Identification Using Secure Multiparty Computation} \\
\hline
\textbf {\large Author(s)} & {\large Julian Bringer \newline Herve Chabanne \newline Alain Patey } \\
\hline
\textbf {\large Abstract/Summary} & {\large This article exhibits an outline of methods of secure calculation to biometric identification. This procedures empower to figure biometric identification while keeping up the protection of biometric information. } \\
\hline
\textbf {\large Problem Solved} & {\large The primary exploration issue speaks the truth keeping up the security of biometric information. These information are basically not recollectable. In the event that the examples are stolen, there is a high plausibility for it being utilized for unlawful exercises. } \\
\hline
\textbf {\large Claimed Contributions} & {\large The creators have expressed that more research still should be made to guarantee security with a more complex separation measures. } \\
\hline
\textbf {\large Related Work} & {\large The prior researches for protecting the privacy of biometric data while maintaining the state of being usable are based on encoding technique. Because of the reversibility, there are no certification of the full security of the biometric information. } \\
\hline
\end{tabular}
\end{document}


